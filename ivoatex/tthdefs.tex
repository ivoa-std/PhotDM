\newif\iftth
\iftth
\input ivoatexmeta
\input tthntbib.sty

\begin{html}
<meta http-equiv="Content-type" content="text/html;charset=UTF-8"/>
\end{html}


\definecolor{ivoacolor}{rgb}{0.0,0.318,0.612}

%%%%%%%%%%%%%%%%%%% ivoatex features

\renewcommand{\author}[2][0]{\def\@tmp{#1}
  \if 0\@tmp
	{\begin{html}<li class="author">\end{html}#2\begin{html}</li>\end{html}}\else
	{\begin{html}<li class="author"><a href="#1">\end{html}#2\begin{html}</a></li>\end{html}}\fi}
\renewcommand{\previousversion}[2][0]{\def\@tmp{#1}
  \if 0\@tmp
	{\begin{html}<li class="previousversion">#2</li>\end{html}}\else
	{\begin{html}<li class="previousversion">
	  <a href="#1">#2</a></li>\end{html}}\fi}
\renewcommand{\ivoagroup}[2][WG]
  {\begin{html}<dd id="ivoagroup" class="#1">#2</dd>\end{html}}
\renewcommand{\editor}[2][0]{\def\@tmp{#1}
  \if 0\@tmp
        {\begin{html}<li class="editor">\end{html}#2\begin{html}</li>\end{html}}\else
        {\begin{html}<li class="editor"><a href="#1">\end{html}#2\begin{html}</a></li>\end{html}}\fi}

\newcommand{\includeMeta}{%
   \special{html:<span id='version'>}\ivoaDocversion\special{html:</span>
   <span id='doctype'>}\ivoaDoctype\special{html:</span>
   <span id='docname'>}\ivoaDocname\special{html:</span>
   <span id='docdate'>}\ivoaDocdate\special{html:</span>}}

% this is for SVN-provided VCS info (and will be picked up by
% tth-ivoa.xslt.
\def\SVN$#1: #2 ${%
	\special{html:<span id='vcs#1'>}#2\special{html:</span>}}

% this is for VCS info put in by git
\def\vcsrevision#1{%
	\special{html:<span id='vcsRev'>}#1\special{html:</span>}}
\def\vcsdate#1{%
	\special{html:<span id='vcsDate'>}#1\special{html:</span>}}
\def\vcsurl#1{%
	\special{html:<span id='vcsURL'>}#1\special{html:</span>}}


\newenvironment{abstract}{%
  \includeMeta
  \begin{html}
    </div> <!-- titlepage -->
    <div id="abstract"><h2>Abstract</h2>
  \end{html}
  }{%
    \special{html:</div> <!-- abstract -->
      <h2>Status of this Document</h2>
      <div id='statusOfThisDocument'>}\ivoaDoctype\special{html:</div>}
    \tableofcontents
  }
   
\newenvironment{generated}{%
  \begin{html}
    <div class="generated">
  \end{html}
  }{%
    \begin{html}
      </div> <!-- generated -->
    \end{html}
  }

\newenvironment{admonition}[1]{%
  \begin{html}
    <div class="admonition">
      <p class="admonition-type">#1</p>
  \end{html}
  }{%
    \begin{html}
      </div> <!-- admonition -->
    \end{html}
  }

\newcommand{\lstinputlisting}[2][plain]{
  \special{html:<div keyvals="#1">}
  \verbatiminput{#2}
  \special{html:</div>}
}
\newcommand{\lstinline}[1]{\special{html:<code>}#1\special{html:</code>}}
\newcommand{\lstloadlanguages}[1]{}
\newcommand{\lstset}[1]{}

\newenvironment{lstlisting}[1][plain]
  {\special{html:<div keyvals="#1">}\tthverbatim{lstlisting}}
  {\special{html:</div>}}
 
\def\nolinkurl#1{\special{html:<span class="nolinkurl">}%
  \verb|#1|%
  \special{html:</span>}}

\newenvironment{bigdescription}{%
    \begin{html}<div class="bigdescription">\end{html}
    \begin{description}\let\term\item
  }{\end{description}\begin{html}</div>\end{html}}
\newenvironment{longtermsdescription}{%
    \begin{html}<div class="longtermsdescription">\end{html}
    \begin{description}
  }{\end{description}\begin{html}</div>\end{html}}

% declare additional css to be included; only effective in the preamble
\renewcommand{\customcss}[1]{%
  \begin{html}<span class="customcss" ref="#1"/>\end{html}}

\newcommand{\specialterm}[2]{%
  \begin{html}<span class="#1">\end{html}#2\begin{html}</span>\end{html}}
\newcommand{\xmlel}[1]{\specialterm{xmlel}{#1}}
\newcommand{\vorent}[1]{\specialterm{vorent}{#1}}
\newcommand{\ucd}[1]{{\sl #1}}

%don't do table rules, these come in through CSS
\newcommand{\sptablerule}{}

\def\dquote{"}

\newcommand{\todo}[2][None]{%
  \begin{html}<span class="redaction">#2</span>\end{html}}

\newenvironment{SCfigure}{\begin{figure}}{\end{figure}}

\newcommand{\ivoatex}{%
  \special{html:<span class="ivoatex">IVOAT<sub>E</sub>X</span>}}

\newcommand{\auxiliaryurl}[1]{%
  \href{https://www.ivoa.net/documents/\ivoaDocname/\ivoaDocdatecode/#1}
  {https://www.ivoa.net/documents/\ivoaDocname/\ivoaDocdatecode/#1}}

\newenvironment{inlinetable}{}{}

% TODO: support some common styles
\newenvironment{compactenum}[1][None]
  {\begin{html}<div class="compact">\end{html}\begin{enumerate}}
  {\end{enumerate}\begin{html}</div>\end{html}}

\newenvironment{compactitem}[1][None]
  {\begin{html}<div class="compact">\end{html}\begin{itemize}}
  {\end{itemize}\begin{html}</div>\end{html}}


%%%%%%%%%%%%%%%%%%%%%%%%%%% Simplified support for Harvard citation style

\newcommand{\harvarditem}[4][0]{%
  \special{html:<a name='#4'/>}
  \if 0#1 \item[#2 (#3)]
  \else \item[#1 (#3)]\fi}
\newcommand{\harvardurl}[1]{\url{#1}}
\newcommand{\harvardand}{\&}
\newcommand{\harvardyearleft}{(}
\newcommand{\harvardyearright}{)}

%%%%%%%%%%%%%%%%%%%%%%%%%%%% Hacks
% shut up harmless error due to some package we're actually using
\def\AtBeginDocument#1{\relax}
\def\pgfsyspdfmark#1#2#3{\relax}
\def\spacefactor{\relax}
\newbox\voidb@x
\def\@m{\relax}

% The default rendering of the captions is really ugly, and so I need to
% fiddle in some HTML to let me style it.  I frankly cannot quite
% say what tth things the first argument to caption would be in
% the following definition; let's consider it an implementation detail.
\newcommand\caption[2][]{\special{html:<div class="caption">}%
  \tthcaption{#2}\special{html:</div>}}

%%%%%%%%%%%%%%%%%%%% support prettyref
% The following is the effective content of the prettyref package,
% minus the default \newrefformat declarations, which are included as examples.
% (required changes: `\@namedef` doesn't work in tth,
% and we need an extra space in `\prettyref`).
%
% Prettyref is by Kevin Ruland kevin@rodin.wustl.edu (1995, 1998)
% and is released to the public domain.
\def\newrefformat#1#2{\expandafter\def\csname pr@#1\endcsname##1{#2}}
\def\prettyref#1{\@prettyref #1:}
\def\@prettyref#1:#2:{\csname pr@#1\endcsname{#1:#2}}
%\newrefformat{eq}{\textup{(\ref{#1})}}
%\newrefformat{lem}{Lemma \ref{#1}}
%\newrefformat{thm}{Theorem \ref{#1}}
%\newrefformat{cha}{Chapter \ref{#1}}
%\newrefformat{sec}{Section \ref{#1}}
%\newrefformat{tab}{Table \ref{#1} on page \pageref{#1}}
%\newrefformat{fig}{Figure \ref{#1} on page \pageref{#1}}

%%%%%%%%%%%%%%%%%%%%%%%%%%%% Now open up titlepage metadata
\begin{html}
  <div id="titlepage">
\end{html}

\fi
